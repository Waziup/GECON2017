\documentclass{beamer}

\usepackage[utf8]{inputenc} 
\usepackage[T1]{fontenc}
\usepackage{lmodern}
\usepackage{graphicx}
\usepackage[english]{babel}
\usepackage{array}
\usepackage{multirow}
\usepackage{caption}
\usepackage{fixltx2e}
\usepackage{listings}
\usepackage{textcomp}
\usepackage[style=authoryear]{biblatex}

\setbeamertemplate{bibliography item}{[\theenumiv]}

\usetheme{Warsaw}

\bibliography{central-bibliography/bibliography}

\begin{document}


\title{Low-cost IoT, Big Data, and Cloud Platform for Developing Countries}
\author{Corentin Dupont, Tomas Bures, Mehdi Sheikhalishahi, Congduc Pham, Abdur Rahim}		 

\institute{FBK/Create-Net\newline cdupont@fbk.eu}


\maketitle

\begin{frame}
  \frametitle{Table of Contents}
  \tableofcontents[]
\end{frame}


\section{Introduction}
\begin{frame}
\frametitle{Introduction}
  
  Current IoT deployments rely heavily on the Cloud... \\
  However, this architecture pose several problems:
  \begin{itemize}
    \item high stress on network and edge infrastructure,
    \item privacy and data ownership,
    \item rural and offshore contexts not well supported.
  \end{itemize}

\end{frame}

\begin{frame}
\frametitle{Introduction}
  
  Fog computing brings new solutions for IoT.
  Examples of containerized IoT functions:
  \begin{itemize}
    \item data stream analysis
    \item prediction and recommendations
    \item data privacy and anonymisation
  \end{itemize}

\end{frame}

\subsection{Overview and contribution}

\begin{frame}
\frametitle{Overview}
 
\end{frame}

\begin{frame}
\frametitle{Contributions}
 
  \emph{Cloud4IoT}: a platform able to containerize IoT functions and optimize their placement.
  \begin{itemize}
    \item deployment of containerized IoT functions on the IoT gateway,
    \item horizontal migration (roaming) of IoT functions,
    \item vertical migration (offloading) of IoT functions.
  \end{itemize}

\end{frame}

\section{State of art}

\begin{frame}
\frametitle{Related works}

 
Characteristics of Fog computing for the Internet of Things (\cite{Bonomi}, \cite{Mulfari2016}):
  \begin{itemize}
    \item low latency,
    \item the location awareness,
    \item widespread geographical distribution of nodes.
  \end{itemize}

\end{frame}

\begin{frame}
\frametitle{Related works}

Containers are particularly indicated for edge computing (\cite{morabito2017, ismail2015}):
  \begin{itemize}
    \item portability,
    \item short start-up time: < 50ms,
    \item small RAM and HD footprint (no kernel redundancy),
    \item app perf on par with native app on Raspberry PI.
  \end{itemize}

\end{frame}

\begin{frame}
\frametitle{Related works}

Cluster management for Edge computing (\cite{pahl2015}):
  \begin{itemize}
    \item Kubernetes
    \item Mesos
  \end{itemize}

Extension of the paper~\cite{pizzolli2016}, with IoT perspective.
  \begin{itemize}
    \item implementation
    \item IoT use cases
    \item related works
    \item future works
  \end{itemize}

\end{frame}



\section{Implementation}

\subsection{IoT function migration}

\begin{frame}
\frametitle{IoT roaming}
 
\end{frame}



\begin{frame}
\frametitle{Discovery and container creation}

  \begin{itemize}
  \item BLE discovers the sensor tag,
  \item orchestrator is notified,
  \item orchestrator deploys the container,
  \item container image is pulled from the Cloud (if not already present on the Gateway),
  \item container is created/attached/connected,
  \item container starts,
  \item first data point is received in the application container.
  \end{itemize}
    
\end{frame}

\begin{frame}
\frametitle{Loss of connection and container destruction}

  \begin{itemize}
    \item BLE detects the loss of connection,
    \item ochestrator is notified,
    \item ochestrator undeploys the container,
    \item container is destroyed.
  \end{itemize}
    
\end{frame}

\subsection{IoT offloading}

\begin{frame}
\frametitle{IoT offloading}

 
\end{frame}


\section{Use cases}

\subsection{IoT roaming for health care}

\begin{frame}
\frametitle{IoT roaming for health care}

  Doctor receives notifications from patient's smart bracelet.
  \begin{itemize}
    \item the patient leaves his/her home and reaches the hospital
    \item the wearable device is automatically associated with a new gateway 
    \item monitoring service is configured accordingly.
  \end{itemize}

The wearables produce data using the open mhealth standard\footnote{http://www.openmhealth.org/}.
    
\end{frame}

\subsection{IoT offloading for remote engine diagnostic}

\begin{frame}
\frametitle{IoT offloading for remote engine diagnostic}
  
  Manufacturing companies deploy continuous and remote monitoring of mechanical engines.
  \begin{itemize}
    \item Diagnostic algorithms run in the gateway during critical phases (test and deployment)
    \item Diagnostic can be pushed back in the Cloud for less critical operations
  \end{itemize}

\end{frame}




\section{Conclusion}

\begin{frame}
\frametitle{Conclusion}
  
  We presented:
  \begin{itemize}
    \item Architecture and implementation of a Fog Computing platform for IoT,
    \item container migration horizontally and vertically,
    \item central Cloud, an Edge and two raspberry PI Gateways,
    \item 2 use cases: IoT roaming in health care context and IoT offloading for remote engine diagnostic.
  \end{itemize}

\end{frame}

\begin{frame}
\frametitle{Future works}

  \begin{itemize}
    \item \emph{Infrastructure as Code} on edge devices such as Raspberry PIs.
    \item tackle the many standards and protocols in different application domains.
    \item security risks and shared devices
  \end{itemize}

\end{frame}

\begin{frame}
\frametitle{Acknowledgements}

The author would like to thank the EU H2020 projects Waziup and Agile, and the Create-Net FBK research centre. \\

\end{frame}

\end{document}
