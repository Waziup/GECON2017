
\section{Introduction}

\todo{change this intro to fit paper objectives: low cost HW review, Security/AuthN/AuthZ, data analytics using Elastic search, market opportunities}
ICT developments in Africa has already enabled significant modernizations across traditional sectors.
Notable examples are the micro-health insurance accessible through mobile devices, index-based crop insurance and crowd-sourced management of public services.
These innovative applications recognize and leverage commonalities between sectors, blur traditional lines, and open up a new field of opportunities.

The opportunity for ICT in Africa is huge especially for IoT and big data: those technologies are promising a big wave of innovation for our daily lives.
The promise of IoT is to connect billions of sensors, devices, equipment and systems.
In turn, the challenge is to drive business outcomes, consumer benefits, and to create new value.
The new mantras for the IoT Era is the collection, convergence and exploitation of data.
The information is collected from sensors, devices, gateways, edge equipment and networks and stored in their respective IoT platforms.
This information is processed in order to increase business efficiency through automation while reducing downtime and improving people productivity.

While developed countries are discussing about massive deployment of IoT, countries in Sub-Saharan Africa are still far from being ready to enjoy the full benefit of Iot.
They face many challenges, such as the lack of infrastructure and the high cost platforms complexity in deployment.
At the same time, it is urgent to promote IoT worldwide : WAZIUP will contribute by reducing part of the technology gap between EU and Africa.
The goal of WAZIUP is to deploy an IoT and big data platform for African needs and validate it through several Sub Saharan Africa real-life use cases.

WAZIUP targets the rural community in Sub-Saharan Africa: about 64\% of the population is living outside cities.
The region will be predominantly rural for at least another generation.
The pace of urbanization here is slower compared to other continents, and the rural population is expected to grow until 2045.
The majority of rural residents live on less than few euros per day.
Rural development is particularly imperative in sub-Saharan Africa, where half of the rural people depends on the agriculture/micro and small farm business, other half faces rare formal employment and pervasive unemployment.
For rural development, technologies have to support several key application sectors such as living quality, health, agriculture and climate changes.

The biggest challenge of WAZIUP is to reduce costs and power consumption while increasing the robustness of the hardware.
Hardware has to be robust enough so as to require lower maintenance and handle environmental and deployment treats as well.
WAZIUP will present an innovative design of the IoT platform dedicated to the rural ecosystem.
To achieve that, low-cost, generic building blocks will be deployed for maximum adaptation to end-applications in the context of the rural economy in developing countries.
Another challenge of WAZIUP is to be able to manage the network deployment and to facilitate IoT communication.
Lower cost solutions has to be used : privilege price and single hop dedicated communication networks, energy autonomous, with low maintenance costs and long lasting operations.
Dynamic management of long range connectivity has to be taken into account (e.g., cope with network \& service fluctuations), such as devices identification, abstraction/virtualization of devices, communication and network resources optimization.
Finally, WAZIUP aim to exploit the potential of big-data applications in the specific rural context.
Data will be collected from the IoT sensors themselves, but WAZIUP will also collect open data from other sources to build predictive models and enrich the platform.

From a technical standpoint, WAZIUP will pay attention to all related privacy and security aspects with specifics addressing the involved communities (farmers, developers).

Continued Openness will be ensured through the release of open specification and open software components and/or algorithms.
Low-cost and low-energy consumption will be possible through the design of innovation hardware (sensors/actuators), and of IoT communication \& network infrastructure.

The challenges outlined above will be tackled using an open IoT-Big Data Platform with affordable sensors connected through an Iot-Cloud open platform. 
The technical functionalities encompassed by the platform will be a cloud-based real-time data collection combined with analytics and automation software, an intelligent analytics of sensor and device data, an integration to third parties platforms and a Platform-as-a-Service (PaaS) provider. 
The PaaS will provide to business clientele with independently maintained platform upon which their web application, services and mobile applications can be built. 

The rest of this book chapter is structured as follows: Section~\ref{sec:platform} presents the architecture of the WAZIUP platform. 
Section~\ref{sec:implem} shows the implementation chosen. 
Section~\ref{sec:deploy} details the deployement of WAZIUP, contrained by its environment.
Section~\ref{sec:usecases} presents four use cases that will be used to validate the WAZIUP concepts.
Section~\ref{sec:relworks} presents a survey of the literature on the topic, together with a survey of the Big Data Open Source tools.
Finally we conclude the chapter in Section~\ref{sec:conclu}.
